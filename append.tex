\chapter{Appendix}

\section{Impulse model}
The qPCR time courses of \tnfa and \sdfonea measured in HPAEC cells are fitted with an impulse model~\cite{Chechik2009}, which is composed of one rising and one falling sigmoidal curve, as defined by
\[
f(x) = \frac{1}{h_1} \cdot \left(h_0+\frac{h_1-h_0}{1+e^{-\beta_1(x-t_1)}}\right) \cdot
\left(h_2+\frac{h_1-h_2}{1+e^{\beta_2(x-t_2)}}\right), 
\]
where the three amplitude (height) parameters determine the initial amplitude 
($h_0$), the peak amplitude
($h_1$), and the steady state amplitude ($h_2$). The onset time $t_1$ is the time of  first transition (inflection point) and the offset $t_2$ is the time of second  transition. Finally, the slope parameters $\beta_1$ and $\beta_2$ determine the slope of the first and second transition. Parameters are estimated using a  Levenberg-Marquardt non-linear least squares  algorithm as implemented in the R package \texttt{minpack.lm}.

\section{Calculating the significantly basal expressed genes}
In order to robustly estimate differentially regulated genes at the basal level and to avoid the commonly encountered 
presence of outliers and skewness~\cite{Marko2011}, we fitted 
a skew-$t$  distribution~\cite{Azzalini2003} to the $\log_2$ raw intensities at 0h. 
In addition to the regular $t$ distribution $f(x)$ with $v_{df}$ degrees of freedom, a skew-$t$ distribution,  $f_{skew}(x)$ has an additional 
skewing parameter $\gamma$ such that  
\[
f_{skew}\left(\frac{x-\xi}{\omega}\right) = 
\begin{cases}
\frac{2\gamma}{\gamma^2+1} \cdot f\left(\gamma\frac{x-\xi}{\omega}\right) & x<0\\
\frac{2\gamma}{\gamma^2+1} \cdot f\left(\frac{1}{\gamma}\frac{x-\xi}{\omega}\right) & x \geqslant 0\\
\end{cases}
\]
where $\xi$  and  $\omega$ denote the location and scale of the distribution.  
The basal level gene expression distribution was fitted using maximum likelihood 
estimation for the 
univariate skew-$t$ distribution, implemented in the R package \texttt{sn}.

\section{Gene set enrichment analysis}
% latex table generated in R 2.15.2 by xtable 1.7-0 package
% Mon Nov 19 15:50:05 2012
\begin{longtable}{rrr}
\caption{Significantly ($p < 0.01$) upregulated gene sets in heterogeneous coculture compared to homogeneous coculture.} \\ 
  \hline
 & p.val & set.size \\ 
  \hline
TIEN\_INTESTINE\_PROBIOTICS\_24HR\_UP & 7.31e-06 & 474 \\ 
   \rowcolor{Gray} SESTO\_RESPONSE\_TO\_UV\_C0 & 2.35e-05 &  98 \\ 
  REACTOME\_CELL\_CYCLE\_MITOTIC & 2.78e-05 & 272 \\ 
   \rowcolor{Gray} REACTOME\_MITOTIC\_M\_M\_G1\_PHASES & 3.61e-05 & 144 \\ 
  HSIAO\_HOUSEKEEPING\_GENES & 4.64e-05 & 356 \\ 
   \rowcolor{Gray} STARK\_PREFRONTAL\_CORTEX\_22Q11\_DELETION\_D & 1.13e-04 & 381 \\ 
  REACTOME\_MITOTIC\_PROMETAPHASE & 1.60e-04 &  82 \\ 
   \rowcolor{Gray} ZHANG\_BREAST\_CANCER\_PROGENITORS\_UP & 2.14e-04 & 331 \\ 
  LI\_WILMS\_TUMOR\_VS\_FETAL\_KIDNEY\_1\_DN & 2.40e-04 & 149 \\ 
   \rowcolor{Gray} OUELLET\_OVARIAN\_CANCER\_INVASIVE\_VS\_LMP\_U & 2.84e-04 & 108 \\ 
  GRAHAM\_CML\_QUIESCENT\_VS\_NORMAL\_QUIESCENT & 3.12e-04 &  76 \\ 
   \rowcolor{Gray} KAUFFMANN\_DNA\_REPAIR\_GENES & 3.32e-04 & 179 \\ 
  PUIFFE\_INVASION\_INHIBITED\_BY\_ASCITES\_UP & 5.04e-04 &  67 \\ 
   \rowcolor{Gray} SHEDDEN\_LUNG\_CANCER\_POOR\_SURVIVAL\_A6 & 5.30e-04 & 389 \\ 
  MISSIAGLIA\_REGULATED\_BY\_METHYLATION\_DN & 5.42e-04 &  87 \\ 
   \rowcolor{Gray} OSMAN\_BLADDER\_CANCER\_UP & 6.26e-04 & 325 \\ 
  KANG\_FLUOROURACIL\_RESISTANCE\_UP & 6.93e-04 &  18 \\ 
   \rowcolor{Gray} REACTOME\_FORMATION\_OF\_ATP\_BY\_CHEMIOSMOTI & 7.68e-04 &  12 \\ 
  REACTOME\_REGULATION\_OF\_GENE\_EXPRESSION\_I & 7.70e-04 &  90 \\ 
   \rowcolor{Gray} REACTOME\_REGULATION\_OF\_BETA\_CELL\_DEVELOP & 8.20e-04 & 100 \\ 
  REACTOME\_FANCONI\_ANEMIA\_PATHWAY & 8.89e-04 &  14 \\ 
   \rowcolor{Gray} REACTOME\_FORMATION\_OF\_A\_POOL\_OF\_FREE\_40S & 9.23e-04 &  85 \\ 
  ROSS\_AML\_OF\_FAB\_M7\_TYPE & 9.46e-04 &  62 \\ 
   \rowcolor{Gray} LENAOUR\_DENDRITIC\_CELL\_MATURATION\_UP & 1.35e-03 &  80 \\ 
  MARTINEZ\_RB1\_AND\_TP53\_TARGETS\_DN & 1.35e-03 & 469 \\ 
   \rowcolor{Gray} LINDGREN\_BLADDER\_CANCER\_CLUSTER\_3\_UP & 1.37e-03 & 273 \\ 
  RUGO\_RESPONSE\_TO\_GAMMA\_RADIATION & 1.38e-03 &  39 \\ 
   \rowcolor{Gray} KYNG\_DNA\_DAMAGE\_BY\_GAMMA\_RADIATION & 1.38e-03 &  39 \\ 
  BARRIER\_CANCER\_RELAPSE\_TUMOR\_SAMPLE\_UP & 1.41e-03 &  14 \\ 
   \rowcolor{Gray} LINDGREN\_BLADDER\_CANCER\_CLUSTER\_1\_DN & 1.50e-03 & 327 \\ 
  SCHWAB\_TARGETS\_OF\_BMYB\_S427G\_DN & 1.51e-03 &  16 \\ 
   \rowcolor{Gray} SCHWAB\_TARGETS\_OF\_BMYB\_I624M\_DN & 1.51e-03 &  16 \\ 
  REACTOME\_CYCLIN\_E\_ASSOCIATED\_EVENTS\_DURI & 1.69e-03 &  54 \\ 
   \rowcolor{Gray} STEIN\_ESRRA\_TARGETS\_RESPONSIVE\_TO\_ESTROG & 1.79e-03 &  36 \\ 
  FLOTHO\_PEDIATRIC\_ALL\_THERAPY\_RESPONSE\_UP & 1.83e-03 &  49 \\ 
   \rowcolor{Gray} FOURNIER\_ACINAR\_DEVELOPMENT\_LATE\_2 & 2.21e-03 & 253 \\ 
  REACTOME\_SCF\_SKP2\_MEDIATED\_DEGRADATION\_O & 2.51e-03 &  48 \\ 
   \rowcolor{Gray} ASTON\_MAJOR\_DEPRESSIVE\_DISORDER\_DN & 2.77e-03 & 140 \\ 
  SENGUPTA\_NASOPHARYNGEAL\_CARCINOMA\_WITH\_L & 2.83e-03 & 300 \\ 
   \rowcolor{Gray} WAKASUGI\_HAVE\_ZNF143\_BINDING\_SITES & 3.21e-03 &  51 \\ 
  ODONNELL\_TARGETS\_OF\_MYC\_AND\_TFRC\_DN & 3.36e-03 &  37 \\ 
   \rowcolor{Gray} VANTVEER\_BREAST\_CANCER\_METASTASIS\_DN & 3.51e-03 &  88 \\ 
  WILCOX\_PRESPONSE\_TO\_ROGESTERONE\_UP & 3.71e-03 & 125 \\ 
   \rowcolor{Gray} CHIANG\_LIVER\_CANCER\_SUBCLASS\_UNANNOTATED & 3.79e-03 &  22 \\ 
  REACTOME\_GENE\_EXPRESSION & 3.85e-03 & 380 \\ 
   \rowcolor{Gray} MUELLER\_COMMON\_TARGETS\_OF\_AML\_FUSIONS\_UP & 3.94e-03 &  12 \\ 
  TAKAO\_RESPONSE\_TO\_UVB\_RADIATION\_UP & 4.24e-03 &  65 \\ 
   \rowcolor{Gray} KEGG\_PROPANOATE\_METABOLISM & 4.38e-03 &  27 \\ 
  BASAKI\_YBX1\_TARGETS\_DN & 4.43e-03 & 271 \\ 
   \rowcolor{Gray} REACTOME\_INSULIN\_SYNTHESIS\_AND\_SECRETION & 4.56e-03 & 117 \\ 
  ROSTY\_CERVICAL\_CANCER\_PROLIFERATION\_CLUS & 4.67e-03 & 128 \\ 
   \rowcolor{Gray} KEGG\_RIBOSOME & 4.95e-03 &  79 \\ 
  REACTOME\_TRANSLATION & 4.96e-03 & 109 \\ 
   \rowcolor{Gray} BERTUCCI\_MEDULLARY\_VS\_DUCTAL\_BREAST\_CANC & 5.16e-03 & 151 \\ 
  KORKOLA\_EMBRYONAL\_CARCINOMA\_UP & 5.17e-03 &  33 \\ 
   \rowcolor{Gray} SMITH\_TERT\_TARGETS\_DN & 5.29e-03 &  66 \\ 
  REACTOME\_SCF\_BETA\_TRCP\_MEDIATED\_DEGRADAT & 5.30e-03 &  44 \\ 
   \rowcolor{Gray} KAYO\_AGING\_MUSCLE\_UP & 5.46e-03 & 152 \\ 
  REACTOME\_SIGNALING\_BY\_WNT & 5.68e-03 &  55 \\ 
   \rowcolor{Gray} MORI\_LARGE\_PRE\_BII\_LYMPHOCYTE\_UP & 5.70e-03 &  51 \\ 
  REACTOME\_PEPTIDE\_CHAIN\_ELONGATION & 5.95e-03 &  76 \\ 
   \rowcolor{Gray} REACTOME\_VIRAL\_MRNA\_TRANSLATION & 5.95e-03 &  76 \\ 
  SPIELMAN\_LYMPHOBLAST\_EUROPEAN\_VS\_ASIAN\_U & 6.09e-03 & 420 \\ 
   \rowcolor{Gray} MARTINEZ\_TP53\_TARGETS\_UP & 6.13e-03 & 485 \\ 
  WALLACE\_PROSTATE\_CANCER\_RACE\_DN & 6.17e-03 &  62 \\ 
   \rowcolor{Gray} NIKOLSKY\_BREAST\_CANCER\_20Q11\_AMPLICON & 6.20e-03 &  27 \\ 
  FLECHNER\_BIOPSY\_KIDNEY\_TRANSPLANT\_REJECT & 6.38e-03 & 484 \\ 
   \rowcolor{Gray} MARTINEZ\_TP53\_TARGETS\_DN & 6.54e-03 & 465 \\ 
  BIOCARTA\_RANMS\_PATHWAY & 6.57e-03 &  10 \\ 
   \rowcolor{Gray} AGUIRRE\_PANCREATIC\_CANCER\_COPY\_NUMBER\_DN & 6.86e-03 & 206 \\ 
  KANG\_IMMORTALIZED\_BY\_TERT\_DN & 6.90e-03 &  91 \\ 
   \rowcolor{Gray} BYSTRYKH\_HEMATOPOIESIS\_STEM\_CELL\_QTL\_CIS & 6.96e-03 &  96 \\ 
  WELCSH\_BRCA1\_TARGETS\_1\_UP & 7.19e-03 & 150 \\ 
   \rowcolor{Gray} MARTORIATI\_MDM4\_TARGETS\_FETAL\_LIVER\_UP & 7.47e-03 &  83 \\ 
  BORCZUK\_MALIGNANT\_MESOTHELIOMA\_UP & 7.47e-03 & 265 \\ 
   \rowcolor{Gray} LE\_EGR2\_TARGETS\_DN & 7.52e-03 &  86 \\ 
  SAKAI\_TUMOR\_INFILTRATING\_MONOCYTES\_UP & 7.80e-03 &  24 \\ 
   \rowcolor{Gray} HADDAD\_T\_LYMPHOCYTE\_AND\_NK\_PROGENITOR\_UP & 7.89e-03 &  67 \\ 
  REACTOME\_FORMATION\_OF\_THE\_TERNARY\_COMPLE & 8.44e-03 &  42 \\ 
   \rowcolor{Gray} BERENJENO\_TRANSFORMED\_BY\_RHOA\_UP & 8.63e-03 & 439 \\ 
  POMEROY\_MEDULLOBLASTOMA\_PROGNOSIS\_DN & 8.72e-03 &  37 \\ 
   \rowcolor{Gray} SENESE\_HDAC1\_TARGETS\_UP & 8.81e-03 & 365 \\ 
  REACTOME\_DIABETES\_PATHWAYS & 9.03e-03 & 333 \\ 
   \rowcolor{Gray} MORI\_EMU\_MYC\_LYMPHOMA\_BY\_ONSET\_TIME\_DN & 9.05e-03 &  16 \\ 
  SANA\_RESPONSE\_TO\_IFNG\_DN & 9.17e-03 &  69 \\ 
   \rowcolor{Gray} DAZARD\_UV\_RESPONSE\_CLUSTER\_G1 & 9.29e-03 &  32 \\ 
  REACTOME\_GTP\_HYDROLYSIS\_AND\_JOINING\_OF\_T & 9.32e-03 &  96 \\ 
   \rowcolor{Gray} LEE\_LIVER\_CANCER\_ACOX1\_UP & 9.46e-03 &  50 \\ 
  CHNG\_MULTIPLE\_MYELOMA\_HYPERPLOID\_DN & 9.50e-03 &  24 \\ 
   \rowcolor{Gray} IIZUKA\_LIVER\_CANCER\_PROGRESSION\_G1\_G2\_DN & 9.57e-03 &  22 \\ 
  BIOCARTA\_BCELLSURVIVAL\_PATHWAY & 9.58e-03 &  11 \\ 
   \rowcolor{Gray}  \hline
\hline
\end{longtable}

% latex table generated in R 2.15.2 by xtable 1.7-0 package
% Mon Nov 19 15:26:24 2012
\begin{longtable}{rrr}
\caption{Significantly ($p < 0.01$) downregulated gene sets in heterogeneous coculture compared to homogeneous coculture.} \\ 
  \hline
 & p.val & set.size \\ 
  \hline
NAGASHIMA\_NRG1\_SIGNALING\_UP & 2.00e-05 & 144 \\ 
   \rowcolor{Gray} LU\_TUMOR\_VASCULATURE\_UP & 3.23e-05 &  29 \\ 
  NAGASHIMA\_EGF\_SIGNALING\_UP & 2.92e-04 &  49 \\ 
   \rowcolor{Gray} MAHADEVAN\_RESPONSE\_TO\_MP470\_DN & 2.98e-04 &  15 \\ 
  AMIT\_EGF\_RESPONSE\_120\_HELA & 3.91e-04 &  52 \\ 
   \rowcolor{Gray} LU\_TUMOR\_ENDOTHELIAL\_MARKERS\_UP & 5.56e-04 &  22 \\ 
  AMIT\_EGF\_RESPONSE\_20\_HELA & 5.75e-04 &  10 \\ 
   \rowcolor{Gray} NIKOLSKY\_BREAST\_CANCER\_22Q13\_AMPLICON & 5.89e-04 &  13 \\ 
  ENGELMANN\_CANCER\_PROGENITORS\_DN & 2.35e-03 &  55 \\ 
   \rowcolor{Gray} NIKOLSKY\_BREAST\_CANCER\_5P15\_AMPLICON & 2.74e-03 &  18 \\ 
  KEGG\_FRUCTOSE\_AND\_MANNOSE\_METABOLISM & 3.82e-03 &  30 \\ 
   \rowcolor{Gray} NICK\_RESPONSE\_TO\_PROC\_TREATMENT\_DN & 4.00e-03 &  25 \\ 
  LANDIS\_ERBB2\_BREAST\_PRENEOPLASTIC\_UP & 4.34e-03 &  15 \\ 
   \rowcolor{Gray} FRIDMAN\_SENESCENCE\_DN & 4.66e-03 &  12 \\ 
  REACTOME\_PEPTIDE\_LIGAND\_BINDING\_RECEPTORS & 5.78e-03 & 130 \\ 
   \rowcolor{Gray} SHEPARD\_CRUSH\_AND\_BURN\_MUTANT\_UP & 7.17e-03 & 122 \\ 
  GENTILE\_UV\_HIGH\_DOSE\_UP & 7.37e-03 &  12 \\ 
   \rowcolor{Gray} SAGIV\_CD24\_TARGETS\_DN & 7.39e-03 &  40 \\ 
  HOFMANN\_MYELODYSPLASTIC\_SYNDROM\_RISK\_UP & 7.76e-03 &  15 \\ 
   \rowcolor{Gray} BERNARD\_PPAPDC1B\_TARGETS\_UP & 8.96e-03 &  28 \\ 
  AMUNDSON\_GAMMA\_RADIATION\_RESISTANCE & 9.51e-03 &  15 \\ 
   \rowcolor{Gray} BIOCARTA\_RANKL\_PATHWAY & 9.64e-03 &  14 \\ 
   \hline
\hline
\end{longtable}

