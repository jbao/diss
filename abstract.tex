\chapter*{Abstract}

\addcontentsline{toc}{chapter}{Abstract} 

Cellular networks 

\chapter*{Zusammenfassung}

\addcontentsline{toc}{chapter}{Zusammenfassung} 

Der Hippocampus spielt eine wichtige Rolle bei sowohl normalen Gehirnfunktionen
als auch epileptischer Aktivit\"at. Hierbei l\"asst sich das Zusammenspiel von  
Struktur und Dynamik wegen der guten Kontrolle \"uber alle Parameter
in einem neuronalen Netzwerkmodell untersuchen. 
In dieser Arbeit wird ein Netzwerkmodell des Gyrus Dentatus pr\"asentiert, das
der Situation in einem akuten Hirnschnittpr\"aparat entspricht. 
Das Modell ist beschr\"ankt durch die 
verf\"ugbaren anatomischen und physiologischen 
Daten in der Literatur. Neurone im Netzwerk werden durch einzelne 
isopotentiale Kompartimente beschrieben,
deren Verhalten der "`leaky integrate-and-fire"' Dynamik mit Leitf\"ahigkeit-%
basierten Synapsen
folgt. Die Neurone sind gleichm\"a\ss ig auf einer ein-dimensionalen 
Struktur verteilt und abh\"angig von ihren Abst\"ande verbunden. 
Die treibenden synaptischen Eing\"ange auf einzelne Neurone werden als unabh\"angige
Poisson-Prozesse modelliert.
In numerischen Simulationen konnte das Netzwerk durch Variation der mittleren
Eingangsraten in 
verschiedene Zust\"ande (asynchron irregul\"ar (AI), synchron irregul\"ar (SI) 
und synchron regul\"ar (SR)) gebracht werden.
Die Aktivierung der hemmenden Interneurone hatte
geringen Einfluss auf die Dynamik des Netzwerks. Von besonderem Interesse
ist der SI-Zustand, in dem eine Oszillation in der Populationsaktivit\"at entstand. 
Die Oszillations%
%frequenz betrug \numprint{7.5} $\pm$ \numprint{1.7}~Hz und lag damit 
im Theta-Frequenzband 
(4--12~Hz). 
Die Oszillation war weitestgehend unabh\"angig von den Anfangsbedingungen
und lie\ss \ sich auf die Konnektivit\"at des Netzwerks 
zur\"uckf\"uhren. Insbesondere f\"uhrte die Verkn\"upfung der Mooszellen mit
den K\"ornerzellen zu einer Subpopulation von K\"ornerzellen, die mit einer
im Vergleich zur Restpopulation deutlich h\"oheren Frequenz und in Phase mit
der Netzwerkoszillation feuerte. Dies unterstreicht die Bedeutung der
Mooszellen, denen auch bei epileptischer Aktivit\"at im Hippocampus eine
wichtige Stellung zukommt.



