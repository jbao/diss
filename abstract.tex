\chapter*{Abstract}

\addcontentsline{toc}{chapter}{Abstract} 

The emergent properties of cellular networks, rather than single molecules, 
are crucial to the external signal processing and the excution of phenotypes
within the cell. In particular, gene regulatory networks serve as a key
driving force behind the long-term cellular fate transitions. To gain more
insight into the design principle of gene regulatory networks, we combined
the time-resolved transcriptome and simulation data to disentangle how the
network topology shapes its dynamics. It is the combination of both topology
and external perturbation that governs the network dynamics, and 
the phenotype is directly related to a subset of
peripheral sparsely connected genes that are strongly regulated. 

Although hidden from the common transcriptome data, 
the upstream events that lead up to the 
transcriptional regulation are also of great interest in studying the 
cellular signal flow. In order to disclose the detailed signal flow pattern,
we applied an integrative approach which infers both the enriched 
transcription factors from the microarray data and the most efficient path
in the protein interaction network based on a random walk motivated distance
metric.

Our theoretical findings are verified in different biological contexts. More
specifically, the key genes identified from their topological and regulatory
features have an impact on the migration of human keratinocytes. 
Cytokine/receptor pairs inferred from the integrative signal flow framework
have also proven to play a role in the communication between lung tumor
and endothelial cells. This work underscores the view of
cellular networks as a complex dynamical system.
It confirms that the marriage between theory
and experiments leads to more biological knowledge.

\chapter*{Zusammenfassung}

\addcontentsline{toc}{chapter}{Zusammenfassung} 

Anstelle von einzelnen Molek\"ulen,
sind es vielmehr die zellul\"aren Interaktionsnetzwerke,
die eine entscheidende Rolle spielen im 
Signalverarbeitungsprozess und in der Etablierung des
Ph\"anotyps. Insbesondere tragen genregulatorische Netzwerke
dazu bei, die \"Anderung der langfristigen zellul\"aren
Zust\"ande zu bestimmen. Um die Gestaltungsprinzipien des
genregulatorischen Netzwerks im
Detail zu verstehen, haben wir das Zusammenspiel von Topologie
und Dynamik im genetischen Netzwerk untersucht anhand von
zeitlich aufgel\"osten Transkriptom und synthetischen
Daten. Mit der Simulation wurde es festgestellt, dass
die Dynamik des Netzwerks durch eine Kombination von
Topologie und St\"orungsfaktor bestimmt ist.
Zudem h\"angt der Ph\"anotyp eng zusammen mit einer Teilmenge
von Genen, die sp\"arlich vernetzt und gleichzeitig
stark reguliert sind. 

Die Funktionsweise der Signalwege, die zur Antwort des
Transkriptoms f\"uhren, l\"asst sich aufgrund der Abtrennung
von der Zeitskala nicht direkt erkennen. Um bestimmte
Muster der Signalleitung nachvollziehen zu k\"onnen, stellen
wir eine 2-Stufige Methode vor, die zuerst aktive
Transkriptionsfaktoren anhand der Microarraydaten ableitet und
dann einen m\"oglichen Signalpfad im 
Protein-Interaktionsnetzwerk mit Hilfe von einem 
Random-Walk-basierten Distanzma\ss \ prognostiziert.

Unsere theoretische Arbeit haben wir in 2 biologischen 
Anwendungsf\"allen \"uberpr\"uft. Zum einen haben wir
anhand der topologischen und dynamischen Eigenschaften
Kandidatengene in der menschlichen Keratinozyten 
identifiziert, die nachweislich unmittelbar f\"ur die
zellul\"are Migration zust\"andig sind. Zum anderen
haben wir mit dem Random-Walk Ansatz 
alle Zytokin-Rezeptor Paare in einem Tumor-
und Endothelzellen Kokultursystem so eingegrenzt,
dass die Ph\"anotyp-relevanten Zytokine mit dem Experiment
festgestellt werden k\"onnen. Diese Arbeit spricht daf\"ur,
die Interaktionsnetzwerke in der Zelle als ein dynamisches
System anzusehen. Ferner wurde es rechtfertigt, dass man
aus der Kombination von Theorie und Experimente mehr 
biologische Kenntnisse gewinnen kann.
