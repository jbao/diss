\chapter*{Abstract}

\addcontentsline{toc}{chapter}{Abstract} 

The emergent properties of cellular networks, rather than single molecules, 
are crucial to the external signal processing and the excution of phenotypes
within the cell. In particular, gene regulatory networks serve as a key
driving force behind the long-term cellular fate transitions. To gain more
insight into the design principle of gene regulatory networks, we combined
the time-resolved transcriptome and simulation data to disentangle how the
network topology shapes its dynamics. It is the combination of both topology
and external perturbation that governs the network dynamics, and 
the phenotype is directly related to a subset of
peripheral sparsely connected genes that are strongly regulated. 

Although hidden from the common transcriptome data, 
the upstream events that lead up to the 
transcriptional regulation are also of great interest in studying the 
cellular signal flow. In order to disclose the detailed signal flow pattern,
we applied an integrative approach which infers both the enriched 
transcription factors from the microarray data and the most efficient path
in the protein interaction network based on a random walk motivated distance
metric.

Our theoretical findings are verified in different biological contexts. More
specifically, the key genes identified from their topological and regulatory
features have an impact on the migration of human keratinocytes. 
Cytokine/receptor pairs inferred from the integrative signal flow framework
have also proven to play a role in the communication between lung tumor
and endothelial cells. This work underscores the view of
cellular networks as a complex dynamical system.
It confirms that the marriage between theory
and experiments leads to more biological knowledge.

\chapter*{Zusammenfassung}

\addcontentsline{toc}{chapter}{Zusammenfassung} 

placeholder


